\documentclass[letterpaper,12pt,openright,oneside]{article}


\usepackage[left=2.5cm,top=2cm,right=2.5cm,nohead,foot=.5cm]{geometry}
\usepackage[spanish,mexico]{babel}
%\usepackage[spanish]{babel}
%\usepackage[latin1]{inputenc}
\usepackage{times}
\usepackage{url}
\usepackage{pdflscape}
\usepackage[utf8]{inputenc}
\usepackage{graphicx}
\usepackage{color}
\usepackage{babelbib}
\usepackage{subfig}
\usepackage{amsmath,amssymb,amsthm} 
% assumes amsmath package installed
%\usepackage[document]{ragged2e}
\usepackage{hyperref}
% \hypersetup{
% colorlinks=true,
% citecolor=blue,
% urlcolor=blue,
% }
\hypersetup{
    colorlinks=true,
    linkcolor=blue,
    filecolor=magenta,      
    urlcolor=cyan,
    pdftitle={SeminarioVI (Enero 2023 - Julio 2023)},
    pdfpagemode=FullScreen,
}

% \urlstyle{rm}
\expandafter\def\expandafter\UrlBreaks\expandafter{\UrlBreaks%  save the current one
  \do\a\do\b\do\c\do\d\do\e\do\f\do\g\do\h\do\i\do\j%
  \do\k\do\l\do\m\do\n\do\o\do\p\do\q\do\r\do\s\do\t%
  \do\u\do\v\do\w\do\x\do\y\do\z\do\A\do\B\do\C\do\D%
  \do\E\do\F\do\G\do\H\do\I\do\J\do\K\do\L\do\M\do\N%
  \do\O\do\P\do\Q\do\R\do\S\do\T\do\U\do\V\do\W\do\X%
  \do\Y\do\Z}
\newcommand{\grad}{$^{\circ}$}
\usepackage{multirow}
\usepackage{todonotes}
\usepackage{makecell}
\usepackage{enumitem}
\usepackage{array}
\usepackage{amsmath}
\usepackage{tabularx}
\usepackage{bm,upgreek}
\usepackage{afterpage}
\usepackage{gensymb}
\usepackage{float}
\usepackage{marvosym}
\usepackage{ragged2e}
\definecolor{light-gray}{gray}{.7}
%\spanishdecimal{.}

%-------------------------------------------
\theoremstyle{plain}
\newtheorem{theorem}{Theorem}
\newcommand{\parcial}[2]{\frac{\partial{#1}}{\partial{#2}}}
\newtheorem{lemma}{Lema}[subsection]
%\newtheorem{proof}[theorem]{Proof}
%\theoremstyle{definition}
\newtheorem{defn}{Definition}
\newtheorem{remark}{Remark}
%\newcommand{\parcial}[2]{\frac{\partial{#1}}{\partial{#2}}}
\DeclareMathOperator{\sign}{sign}
\newcommand{\m}[1]{\mathbf{#1}}
\DeclareMathOperator{\diag}{diag}

\renewcommand{\baselinestretch}{1.15}

\newcolumntype{P}[1]{>{\centering\arraybackslash}p{#1}}
\setlength {\marginparwidth }{2cm} 
% ####################################################
\usepackage{lineno} %paquete para agregar números a cada linea
\urlstyle{same}
% ####################################################
\begin{document}

%\pagestyle{empty} %No headings for the first pages.


%\includegraphics[width=0.05\textwidth]{./logo.jpg}~\\[1cm]

\begin{titlepage}
\begin{center}

% Upper part of the page. The '~' is needed because \\
% only works if a paragraph has started.


\textsc{\LARGE Universidad Autónoma de Querétaro}\\[1 cm]
\includegraphics[width=0.15\textwidth]{./uaqlogo}~\\[0.75cm]
\textsc{\Large Facultad de Ingeniería \\ \large{Doctorado en Ingeniería}}\\[0.5 cm]
\textsc{\large Seminario VI \\ (Enero 2023 - Julio 2023)}\\[0.5cm]
%\small{Maestría en Instrumentación y Control Automático \\ 

% Title
%\HRule \\[0.4cm]
{ 

\huge \bfseries 

GUÍA PMBOK
\\[0.4cm] }
%\vspace*{2cm}
\vspace*{1.0cm}

% Author and supervisor
\large \textbf{Presentan:} \\
\large \textbf{} Gendry Alfonso Francia\\\vspace*{0.5cm}
Ireri Graciela Segura Gutierrez\\\vspace*{0.5cm}
\large \textbf{} Julio Alberto Ramírez Montañez\\\vspace*{0.5cm}
\large \textbf{} Luis Alberto Ángeles Hurtado\\\vspace*{0.5cm}

\textbf{Profesor:} \\ Dr. Juan Carlos Jáuregui $^{1}$\\\vspace*{0.5cm}
\raggedright
$^{1}$ Universidad Autónoma de Querétaro, Facultad de Ingeniería.\\
%$^{2}$ Instituto Tecnológico de Celaya, Departamento de Ingeniería Electrónica.
\vfill
\centering
% Bottom of the page
{\large \today}
%{\large 5 de diciembre de 2020}
\end{center}
\end{titlepage}

\newpage
%%%%
\hypersetup{linkcolor=black}
\tableofcontents
\listoffigures
% \listoftables

\newpage
\linenumbers
% 
%
\addcontentsline{toc}{section}{Introducción}
\section*{Introducción}

Un proyecto es un esfuerzo para lograr un objetivo específico mediante una serie especial de actividades interrelacionadas y la utilización adecuada de los recursos.
Los proyectos posibilitan la identificación, análisis y selección de alternativas de solución adecuadas a necesidades o problemas específicos.
Permite y aporta la toma de decisiones, recomendando soluciones técnicas más adecuadas.
Los proyectos tienen beneficios porque gestionan mejor los recursos disponibles de la organización.
Se controla y verifica el objetivo definido.
Se tiene claramente definido el alcance en tiempo y costo y reduce los riesgos sociales, económicos.
La administración de proyectos es el conjunto de actividades y herramientas necesarias para guiar un proyecto a un resultado exitoso.
Para el PMBOK la administración de proyectos es la aplicación de conocimientos, habilidades, herramientas y técnicas a las actividades del proyecto para cumplir con los requisitos del mismo.
% 
% 
% ¿Qué voy a hacer?, ¿Cómo lo voy a hacer? y ¿Para qué lo voy a hacer?
\addcontentsline{toc}{section}{Objetivo general}
\section*{Objetivo general}

Desarrollar un documento que permita a los estudiantes del programa doctoral entender la administración de su proyecto desde el punto de vista del PMBOK para implementarlo en sus trabajos de investigación. 
% 
%
\addcontentsline{toc}{subsection}{Objetivos específicos}
\subsection*{Objetivos específicos}

\begin{itemize}
    \item Resumir la sexta versión de PMBOK para adaptarlo a un proyecto de investigación.
    \item Desarrollar un documentos con ciertas características que guíen a los alumnos del doctorado con la administración de sus proyectos.
\end{itemize}

% \addcontentsline{toc}{section}{4.1 Develop Project Charter}
% \section*{4.1 Develop Project Charter}
% \addcontentsline{toc}{section}{4.1 Desarrollo de la carta del proyecto}
% \section*{4.1 Desarrollo de la carta del proyecto}
\addcontentsline{toc}{section}{Desarrollo de la carta del proyecto}
\section*{Desarrollo de la carta del proyecto}

% .1 Inputs
%  .1 Business documents
%  .2 Agreements
%  .3 Enterprise environmental
%  factors
%  .4 Organizational process
%  assets
% \addcontentsline{toc}{section}{4.1.1 Inputs}
% \section*{4.1.1 Inputs}
% \addcontentsline{toc}{section}{4.1.1 Entradas}
% \section*{4.1.1 Entradas}
\addcontentsline{toc}{section}{Entradas}
\section*{Entradas}
% 
% 
% \addcontentsline{toc}{subsection}{4.1.1.1 Business documents}
% \subsection*{4.1.1.1 Business documents}
% \addcontentsline{toc}{subsection}{4.1.1.1 Documentos de Negocios}
\addcontentsline{toc}{subsection}{Documentos de Negocios}

Una carta de proyecto es un documento firmado que establece formalmente el alcance de un proyecto, deja en claro las metas y los objetivos del proyecto a los diferentes grupos o personas que están involucradas y que tienen un interés en común.

% \subsection*{4.1.1.1 Documentos de Negocios}
\subsection*{Documentos de Negocios}

En esta parte se debe describir la información necesaria desde un punto de vista empresarial para determinar si los resultados esperados del proyecto justifican la inversión requerida. Por lo general, la necesidad empresarial y el análisis de costo-beneficio están contenidos en el caso de negocio para justificar y establecer límites para el proyecto. El caso de estudio debe contener: demanda en el mercado, necesidades organizacionales, avances tecnológicos, requerimientos del consumidor, requerimientos legales, impacto ecológico y el impacto social. 
El principal objetivo al empezar a armar el caso de estudio es definir la viabilidad económica en el documento, esto será utilizado para establecer la validez de los beneficios de un componente seleccionado que carece de una definición suficiente y que se utiliza como base para la autorización de otras actividades de gestión de proyectos. Es importante enumerar los objetivos y razones para el inicio del proyecto, con el objetivo de medir el éxito del proyecto al final del mismo.

% 
% 
% \addcontentsline{toc}{subsection}{4.1.1.2 Agreements}
% \subsection*{4.1.1.2 Agreements}
% \addcontentsline{toc}{subsection}{4.1.1.2 Acuerdos}
% \subsection*{4.1.1.2 Acuerdos}
\addcontentsline{toc}{subsection}{Acuerdos}
\subsection*{Acuerdos}

Una vez comenzado el documento es importante definir los acuerdos para delimitar las obligaciones de las partes y los objetivos que se quieren alcanzar.
Los acuerdos pueden tomar la forma de contratos, memorandos de entendimiento (MOU), acuerdos de nivel de servicio (SLA), cartas de acuerdo, cartas de intención, acuerdos verbales, correo electrónico u otros acuerdos escritos. Normalmente, un contrato se utiliza cuando se realiza un proyecto para un cliente externo.
Un contrato es un acuerdo mutuamente vinculante que obliga al vendedor a proporcionar los productos, servicios o resultados especificados; obliga al comprador a indemnizar al vendedor; y representa una relación jurídica que está sujeta a recurso en los tribunales. Los componentes principales en un documento de acuerdo variarán y pueden incluir, entre otros: 

\begin{itemize}
    \item Declaración de trabajo en materia de adquisiciones o entregables principales;
    \item Cronograma, hitos o fecha en la que se requiere una programación;
    \item Presentación de informes sobre el rendimiento;
    \item Precios y condiciones de pago;
    \item Criterios de inspección, calidad y aceptación;
    \item Garantía y soporte futuro del producto;
    \item Incentivos y sanciones;
    \item Seguros y fianzas de cumplimiento;
    \item Aprobaciones de subcontratistas subordinados;
    \item Términos y condiciones generales;
    \item Gestión de solicitudes de cambio; y
    \item Cláusula de rescisión y mecanismos alternativos de resolución de conflictos.
    % \item \href{https://www.uaq.mx/index.php/conocenos/rectoria/67-conocenos/350-proyectos-institucionales}{Proyectos-Institucionales}, \href{https://derechosuniversitarios.uaq.mx/}{Derechos Universitarios}.
    % \item \url{https://www.uaq.mx/index.php/conocenos/rectoria/67-conocenos/350-proyectos-institucionales}, \url{https://derechosuniversitarios.uaq.mx/}.
    % \item \href{run:https://www.uaq.mx/leyes/regadqu.pdf}{Capitulo II de la planeacion, programacion, presupuestacion y evaluacion de la administracion de los recursos materiales. capitulo VIII del padron de proveedores de la universidad capitulo X de los almacenes.pdf}
\end{itemize}

Para ello se pueden consultar los siguientes documentos, donde se especifican los acuerdos hechos dentro de las instalaciones de la Universidad y que es primordial tomarlos en cuanta al realizar la Carta del Proyecto.
Uno de ellos es el repositorio de los proyectos instituciones que existen dentro de la universidad y con los cuales puede haber una colaboración. Se puede consultar estos proyectos a través de la siguiente liga: \url{https://www.uaq.mx/index.php/conocenos/rectoria/67-conocenos/350-proyectos-institucionales}
Por otra parte, es importante también consultar los estatutos de los derechos universitarios publicado en la siguiente liga, con el fin de que se siga la normativa vigente dentro de la universidad, la cual nos permite realizar colaboraciones lo más armoniosa posible \url{https://derechosuniversitarios.uaq.mx/}.
Por último, consultar los siguientes capítulos para la planeación de proyectos y sus colaboraciones dentro de la universidad, para el manejo de recursos materiales necesarios para llevar a cabo el proyecto. Además, también es importante consultar el padrón de proveedores de la universidad para adquirir los recursos materiales, y en caso de no estar dentro de dicho patrón, consultar los requerimientos para poder ser dados de alta \href{run:https://www.uaq.mx/leyes/regadqu.pdf}{Capitulo II de la planeacion, programacion, presupuestacion y evaluacion de la administracion de los recursos materiales. capitulo VIII del padron de proveedores de la universidad capitulo X de los almacenes.pdf}.


% 
% 
% \addcontentsline{toc}{subsection}{4.1.1.3 Enterprise environmental factors}
% \subsection*{4.1.1.3 Enterprise environmental factors}
% \addcontentsline{toc}{subsection}{4.1.1.3 Factores Ambientales Empresariales}
% \subsection*{4.1.1.3 Factores Ambientales Empresariales}
\addcontentsline{toc}{subsection}{Factores Ambientales Empresariales}
\subsection*{Factores Ambientales Empresariales}

Los factores ambientales que pueden influir al armar el documento son los siguientes:

\begin{itemize}
    \item Estándares gubernamentales o de la industria (calidad, seguridad, entre otros),
    \item Leyes, restricciones y/o normativa vigente,
    \item Condiciones del mercado,
    \item Cultura y Política,
    \item Marco organizacional,
    \item Expectativas de las partes interesadas y umbrales de riesgo.
\end{itemize}
% 
% 
% \addcontentsline{toc}{subsection}{4.1.1.4 Organizational process assets}
% \subsection*{4.1.1.4 Organizational process assets}
% \addcontentsline{toc}{subsection}{4.1.1.4 Activos de Procesos Organizativos}
% \subsection*{4.1.1.4 Activos de Procesos Organizativos}
\addcontentsline{toc}{subsection}{Activos de Procesos Organizativos}
\subsection*{Activos de Procesos Organizativos}

Los activos del proceso organizativo que pueden influir en el proceso de Elaboración de la Carta del Proyecto incluyen, entre otros, pero no limitados a:

\begin{itemize}
    \item \textbf{Políticas, procesos y procedimientos estándar de la organización}: Se refiere a las directrices, marcos y metodologías establecidos que una organización utiliza para iniciar y gestionar proyectos. Son importantes para garantizar que la carta del proyecto se elabora de forma coherente y estructurada, y que se ajusta a las metas y objetivos de la organización.
    \item \textbf{Marco de gobierno de carteras, programas y proyectos} (funciones y procesos de gobierno para proporcionar orientación y toma de decisiones): Garantizan que los proyectos se ajustan a las metas y objetivos generales de la organización y se gestionan de forma coherente y estructurada pueden influir en la forma en que se elabora, revisa y aprueba la carta del proyecto, así como en los criterios utilizados para evaluar su integridad y calidad.
        \begin{itemize}
            \item o	En el caso de la UAQ, la guía para el registro, informe de avances y cambios de los proyectos de investigación puede ser accedida en: \href{run:https://dip.uaq.mx/docs/investigacion/formatos/A-50-01-Guia-para-el-registro-informes-y-cambios-a-los-proyectos-de-investigacion_v1.2.pdf}{Guia-para-el-registro-informes-y-cambios-a-los-proyectos-de-investigación.pdf}.
        \end{itemize}
    \item \textbf{Métodos de seguimiento y elaboración de informes}: Ayudar a las partes interesadas a mantenerse informadas sobre el progreso del proceso de elaboración de la Carta.
    \item \textbf{Plantillas} (por ejemplo, plantilla de carta de proyecto);
        \begin{itemize}
            \item Para una descripción general, descargables, guía para registros y acceso al master web de calidad, consultar la siguiente liga: \url{https://dip.uaq.mx/index.php/investigacion/coordinacion-de-investigacion-y-enlace}
            \item o	Para proyectos de vinculación, Sistema Universitario de Vinculación (SUV), consultar la siguiente liga: \url{https://www.uaq.mx/svsu/index.php/sistema-universitario-de-vinculacion}
            \item o	Para proyectos en su modalidad de extensión, consultar la siguiente liga: \url{https://extension.uaq.mx/index.php/convocatoria} 
        \end{itemize}
    \item \textbf{Repositorio de información histórica y lecciones aprendidas} (p. ej., registros y documentos del proyecto, información sobre los resultados de anteriores decisiones de selección de proyectos, e información sobre el rendimiento de anteriores proyectos)
\end{itemize}

%  
%  
% .2 Tools & Techniques
%  .1 Expert judgment
%  .2 Data gathering
%  .3 Interpersonal and team
%  skills
%  .4 Meetings
% \addcontentsline{toc}{section}{4.1.2 Tools \& Techniques}
% \section*{4.1.2 Tools \& Techniques}
% \addcontentsline{toc}{section}{4.1.2 Técnicas y Herramientas}
% \section*{4.1.2 Técnicas y Herramientas}
\addcontentsline{toc}{section}{Técnicas y Herramientas}
\section*{Técnicas y Herramientas}

% 
%
% \addcontentsline{toc}{subsection}{4.1.2.1 Expert judgment}
% \subsection*{4.1.2.1 Expert judgment}
% \addcontentsline{toc}{subsection}{4.1.2.1 Criterio/Juicio Experto}
% \subsection*{4.1.2.1 Criterio/Juicio Experto}
\addcontentsline{toc}{subsection}{Criterio/Juicio Experto}
\subsection*{Criterio/Juicio Experto}

El juicio experto se define como el juicio basado en la experiencia en un área de aplicación, área de conocimiento, disciplina, industria, etc., según corresponda a la actividad que se esté realizando. Dicha pericia puede ser aportada por cualquier grupo o persona con formación especializada, conocimientos, habilidades, experiencia o formación.
Para este proceso, se debe considerar la experiencia de personas o grupos con conocimientos especializados o capacitación en los siguientes temas:

\begin{itemize}
    \item Estrategia Organizacional.
    \item Gestión de Beneficios.
    \item Conocimiento técnico de la industria y área de enfoque del proyecto.
    \item Estimación de presupuesto y duración.
    \item Identificación de riesgos.
\end{itemize}
% 
% 
% \addcontentsline{toc}{subsection}{4.1.2.2 Data gathering}
% \subsection*{4.1.2.2 Data gathering}
% \addcontentsline{toc}{subsection}{4.1.2.2 Recopilación de datos}
% \subsection*{4.1.2.2 Recopilación de datos}
\addcontentsline{toc}{subsection}{Recopilación de datos}
\subsection*{Recopilación de datos}

Las técnicas de recopilación de datos que se pueden utilizar para el proceso incluyen, entre otras, las siguientes:

\begin{itemize}
    \item \textbf{Lluvia de ideas}: Esta técnica se utiliza para identificar una lista de ideas en un corto período de tiempo. Se lleva a cabo en un ambiente de grupo y está dirigido por un facilitador. La lluvia de ideas consta de dos partes: generación de ideas y análisis. La lluvia de ideas se puede utilizar para recopilar datos y soluciones o ideas de las partes interesadas, los expertos en la materia y los miembros del equipo al desarrollar el acta de constitución del proyecto.
    \item \textbf{Grupos de enfoque}: Reúnen a las partes interesadas precalificadas y expertos en la materia para aprender sobre el riesgo percibido del proyecto, los criterios de éxito, expectativas y actitudes sobre un producto, servicio o resultado propuesto. Es guiado por un moderador capacitado para llevar la discusión de una manera más conversacional que una entrevista individual.
    \item \textbf{Entrevistas}: Se utilizan para obtener información sobre requisitos, suposiciones o restricciones de alto nivel, criterios de aprobación y otra información de las partes interesadas hablando directamente con ellas. Entrevistar a participantes de proyectos experimentados, patrocinadores, otros ejecutivos y expertos en la materia puede ayudar a identificar y definir las características y funciones de los productos deseados.
\end{itemize}
% 
% 
% \addcontentsline{toc}{subsection}{4.1.2.3 Interpersonal and team skills}
% \subsection*{4.1.2.3 Interpersonal and team skills}
% \addcontentsline{toc}{subsection}{4.1.2.3 Habilidades Interpersonales y de Equipo}
% \subsection*{4.1.2.3 Habilidades Interpersonales y de Equipo}
\addcontentsline{toc}{subsection}{Habilidades Interpersonales y de Equipo}
\subsection*{Habilidades Interpersonales y de Equipo}

Estas incluyen la gestión de conflictos, la facilitación y la gestión de reuniones.
\textbf{La gestión de conflictos} es importante debido a que durante el desarrollo de un proyecto siempre esta expuesto a problemas, desde escases de recursos hasta en la diferencia de estilos de trabajo de los integrantes. 
Hablando desde las habilidades interpersonales del equipo es necesario establecer ciertas reglas de trabajo, roles definidos, y líneas de comunicación, con la finalidad de reducir los conflictos internos y aprovechar las fortalezas de cada miembro. 
Entre las reglas pueden definirse los horarios de trabajo, el lugar del mismo, las acciones correctivas si se rompe alguna regla; mientras que en los roles se tiene que distinguir claramente la jerarquía, no todos pueden el líder de equipo, pero si pueden compartir responsabilidades, y en la línea de comunicación tiene que ser eficiente, si ocurre un problema este tiene que ser expuesto ante el grupo o los integrantes con mayor jerarquía para buscar soluciones.
\textbf{La facilitación} es desarrollada por un miembro del equipo, el cual tendrá la responsabilidad de guiar al grupo, ayudando al entendimiento mutuo de los integrantes, favoreciendo que todos los integrantes entiendan los objetivos a alcanzar en el proyecto, y sus responsabilidades a desarrollar.
En un proyecto de investigación este rol lo toma el director de tesis o el responsable principal del trabajo de investigación.
\textbf{La gestión de reuniones} consiste en tomar medidas apropiadas para hacer que las reuniones sean usadas de manera eficaz, en lugar de perder tiempo; 
estas serán un acuerdo mutuo entre el líder del equipo y los demás integrantes (director de tesis, sínodo, coordinador del programa, responsable del proyecto y becarios). (Definido por las normas complementarias 2018 en la sección 2.4 Coordinadores de programas educativos de posgrado)
Para la planificación de las reuniones debe de considerarse estos pasos:

\begin{itemize}
    \item Preparar la agenda, estableciendo los objetivos de la reunión.
    \item Asegurar un horario establecido, el cual no deberá de ser excedido.
    \item Asegurarse que todos los miembros relevantes estén invitados, cada miembro del equipo tendrá una responsabilidad distinta, y no siempre será necesario que participen en todas las reuniones.
    \item Registrar las acciones realizadas, con la finalidad de tener un historial, permitiendo medir los avances obtenidos entre reuniones.
\end{itemize}

% 
% 
% \addcontentsline{toc}{subsection}{4.1.2.4 Meetings}
% \subsection*{4.1.2.4 Meetings}
% \addcontentsline{toc}{subsection}{4.1.2.4 Reuniones}
% \subsection*{4.1.2.4 Reuniones}
\addcontentsline{toc}{subsection}{Reuniones}
\subsection*{Reuniones}

Son elementos importantes utilizados para analizar el enfoque del proyecto, sus avances, los problemas presentados hasta el momento, con la finalidad de verificar el alcance de los objetivos expuestos desde el inicio del proyecto.
La principal reunión a tener en cuenta es la reunión de lanzamiento, cuyo propósito es comunicar los objetivos del proyecto, su delimitación, los roles y responsabilidades de cada integrante del equipo.
(Definido por las normas complementarias 2018 en la sección 2.4 Coordinadores de programas educativos de posgrado).
Es recomendable realizar una reunión al iniciar cada etapa del proyecto y varias intermedias con la finalidad de monitorear el avance obtenido y el desarrollo que se pretende alcanzar en la próxima etapa.

% 
%
% .3 Outputs
%  .1 Project charter
%  .2 Assumption log
% \addcontentsline{toc}{section}{4.1.3 DEVELOP PROJECT CHARTER: OUTPUTS}
% \section*{4.1.3 DEVELOP PROJECT CHARTER: OUTPUTS}
% \addcontentsline{toc}{section}{4.1.3 Desarrollo de la carta del proyecto: Resultados}
% \section*{4.1.3 Desarrollo de la carta del proyecto: Resultados}
\addcontentsline{toc}{section}{Desarrollo de la carta del proyecto: Resultados}
\section*{Desarrollo de la carta del proyecto: Resultados}

% 
%
% \addcontentsline{toc}{subsection}{4.1.3.1 Project charter}
% \subsection*{4.1.3.1 Project charter}
% \addcontentsline{toc}{subsection}{4.1.3.1 Acta de Constitución del Proyecto}
% \subsection*{4.1.3.1 Carta de Constitución del Proyecto}
\addcontentsline{toc}{subsection}{Acta de Constitución del Proyecto}
\subsection*{Carta de Constitución del Proyecto}

Crear un documento con las siguientes características.

\begin{itemize}
    \item El propósito o la justificación del proyecto, 
        \begin{itemize}
            \item Detallar la información clara y específica dentro de la autorización del proyecto, 
            \item Puede ser complementada con información adicional como parte de la estructura de la solicitud incluyendo conceptos.
        \end{itemize}
    \item Los objetivos medibles del proyecto y los criterios de éxito asociados,
        \begin{itemize}
            \item El objetivo debe de ser claro con la meta o finalidad a cumplir para la que se disponen medios determinados.
            \item Debe ser medible, cuantificable, alcanzable, específicos en el tiempo y mensurados.
        \end{itemize}
    \item Los requisitos de riesgo,
        \begin{itemize}
            \item Establecer los requerimientos que ponen en alto riesgo el inicio del proyecto.
            \item Cumplir con los requisitos normativos, ambientales, legales o sociales.
            \item Satisfacer las solicitudes o necesidades de las partes interesadas.
        \end{itemize}
    \item La descripción de alto riesgo del proyecto y sus límites,
        \begin{itemize}
            \item Establecer los límites de las políticas, protocolos y procedimientos organizacionales en lugar de esperar a que se les otorgue.
            \item Definir la necesidad comercial y el análisis de costo-beneficio para justificar y establecer los límites del proyecto.
        \end{itemize}
    \item Los supuestos y las restricciones,
        \begin{itemize}
            \item Planificar la Gestión de Riesgos,
            \item Identificar riesgos,
            \item Realizar análisis cualitativo de riesgos,
            \item Realizar análisis de riesgos cuantitativos,
            \item Planificar las respuestas a los riesgos.
        \end{itemize}
    \item El resumen del cronograma de hitos,
        \begin{itemize}
            \item Determinar los eventos significativos en el programa maestro.
            \item Representar el cronograma en forma de (1) gráfico de hitos, (2) un cronograma resumido como un gráfico de barras y (3) un cronograma detallado como un diagrama de gráfico de barras vinculado al cronograma del proyecto.
        \end{itemize}
    \item Recursos financieros preaprobados,
        \begin{itemize}
            \item Describir los recursos financieros a partir de los cuales se desarrollan los costos detallados del proyecto. 
            \item Definir los requisitos de aprobación del proyecto que influirán en la gestión de los costos del proyecto.
        \end{itemize}
    \item La lista de interesados clave,
    \item Los requisitos de aprobación del proyecto (es decir, en qué consiste el éxito del proyecto, quién decide si el proyecto tiene éxito y quién firma la aprobación del proyecto),
    \item Criterios de salida del proyecto (es decir, cuáles son las condiciones que se deben cumplir para cerrar o cancelar el proyecto o fase),
    \item El director del proyecto asignado, su responsabilidad y su nivel de autoridad
        \begin{itemize}
            \item Describir su responsabilidad en el proyecto y lo necesario para mantener una buena comunicación.
        \end{itemize}
    \item El nombre y el nivel de autoridad del patrocinador o de quienes autorizan el acta de constitución del proyecto.
\end{itemize}

\textbf{Se recomienda realizar el llenado de la carta del proyecto al inicio del proyecto y actualizarla en dado caso de conseguir financiamiento de algún patrocinador así como llevar un control de cambios de la carta.}
Verificar que el formato que se proponga también contemple los requisitos de la Secretaría de Extensión para proyectos vinculados

Se propone el formato, véase la Figura \ref{CartaConstitutiva} o descargue el formato en la siguiente liga: \url{https://github.com/LAAngeles/Seminario_VI_2023-1/blob/main/FormatoCartaDelProyecto/formatoCartaDelProyecto.docx}
% 
% 
%#############################################
\begin{figure}[h]
    \centering
    \includegraphics[trim = {18mm 18mm 12mm 11mm},clip,scale=0.5]{img/formatoCartaDelProyecto.pdf}
    \caption{Ejemplo del formato de la carta constitutiva del proyecto.}
    \label{CartaConstitutiva}
\end{figure}
%#############################################
% 
% 
% \addcontentsline{toc}{subsection}{4.1.3.2 Assumption log}
% \subsection*{4.1.3.2 Assumption log}
% \addcontentsline{toc}{subsection}{4.1.3.2 Registro de supuestos}
% \subsection*{4.1.3.2 Registro de supuestos}
\addcontentsline{toc}{subsection}{Registro de supuestos}
\subsection*{Registro de supuestos}

Desarrollar un registro de supuestos más detallado para registrar todos los supuestos y restricciones a lo largo del ciclo de vida del proyecto.
% Las suposiciones y restricciones estratégicas y operativas de alto nivel normalmente se identifican en el caso de negocios antes de que se inicie el proyecto y fluirán hacia el acta de constitución del proyecto. 
Los supuestos de actividades y tareas de nivel inferior que se generan a lo largo del proyecto, como la definición de especificaciones técnicas, estimaciones, el cronograma, los riesgos, etc. 
% 
% 
\addcontentsline{toc}{section}{Gestión del alcance del proyecto}
\section*{Gestión del alcance del proyecto}

La gestión del alcance del proyecto se ocupa principalmente de
definir y controlar lo que está y no está incluido en el proyecto.

% 5.4 Crear WBS—El proceso de subdividir los entregables del proyecto y el trabajo del proyecto en partes más pequeñas y manejables componentes

La ``WBS'' es una técnica de descomposición jerárquica del trabajo que divide el alcance del proyecto en partes manejables, lo que facilita la planificación y gestión del proyecto. 
El trabajo planificado está contenido dentro del nivel más bajo de componentes de la WBS, que se denominan paquetes de trabajo. 
La WBS es una herramienta importante para la gestión de recursos, ya que permite identificar los recursos necesarios para cada paquete de trabajo y para todo el proyecto. 
El plan de gestión de proyectos se define, de manera general y resumida, cómo se ejecuta, supervisa, controla y cierra el proyecto.
% 
% 
\addcontentsline{toc}{subsection}{Entradas}
\subsection*{Entradas}

En esta sección se describen las primeras etapas del proyecto, definidas por documentos entregables como lo son:

\begin{itemize}
    \item Plan para la dirección del proyecto. Es la gestión del proyecto que documenta como será creado el proyecto, la WBS a partir del enunciado del alcance del proyecto.
    \item Documentos del proyecto.
        \begin{itemize}
            \item Enunciado del alcance del proyecto. Describe el alcance del proyecto del trabajo que será realizado y el trabajo excluido.
            \item Documentos de requisitos. Detallan los requisitos que deben cumplirse para el desarrollo del proyecto.
        \end{itemize}
    \item Factores ambientales de la empresa.
    \item Activos de procesos de la organización.
    
\end{itemize}
% 
% 
\addcontentsline{toc}{subsection}{Herramientas y técnicas}
\subsection*{Herramientas y técnicas}

\begin{itemize}
    \item Juicios de expertos. Se requiere equipos de trabajo o la pericia individual en la experiencia con proyectos similares.
    \item Descomposición. Técnica utilizada para dividir y subdividir el alcance y los resultados del proyecto en partes más pequeñas y manejables. Generalmente envuelve las siguientes actividades:
        \begin{itemize}
            \item Identificar y analizar los entregables y el trabajo relacionado.
            \item Estructuración y organización del WBS.
            \item Descomposición de los niveles superiores de WBS en componentes detallados de nivel inferior.
            \item Desarrollo y asignación de códigos de identificación a los componentes de WBS.
            \item Comprobación de que el grado de descomposición de los entregables es el adecuado.
        \end{itemize}
\end{itemize}
% 
% 
\addcontentsline{toc}{subsection}{Salidas}
\subsection*{Salidas}

\begin{itemize}
    \item Línea base de referencia, incluye:
        \begin{itemize}
            \item Declaración sobre el alcance del proyecto.
            \item WBS.
            \item Programa de trabajo.
            \item Programa de planificación.
            \item Diccionario WBS.
        \end{itemize}
    \item Actualización de los documentos del proyecto.
        \begin{itemize}
            \item Registro de asunciones.
            \item Documentación de requisitos.
        \end{itemize}
\end{itemize}

% La WBS es una técnica para desglosar el proyecto en partes pequeñas, permitiendo un mejor desarrollo del mismo y una gestión apropiada de los recursos.


\end{document}
